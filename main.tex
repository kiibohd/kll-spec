\documentclass{kiibohd-template}

\begin{document}


%% General Notes and Guidelines when editing this document %% XXX <---
 %  - To say this calmly, ONE SENTENCE PER LINE, no exceptions.
 %  - If you want to use a table, there is a table macro
 %  - If you want to include a diagram/figure/picture, there is a macro for that
 %  - Use \todo and \todo[inline] to indicate TODOs
 %  - Use LaTex comments to indicate formatting and tricky LaTex usage
 %  - XXX indicates something to take note of (it highlights in many text editors, as does TODO)
 %  - FIXME is another highlighting tag :D


\title{KLL Spec v0.5.6}
\titlename{Keyboard Layout Language Spec}
\author{HaaTa - Jacob Alexander}
\email{haata@kiibohd.com}
\docrev{DRAFT - 0.5.6}
\urlname{kiibohd.com}

%%% Titlepage %%%
\maketitle


%%% Revision History %%%
\begin{versionhistory}
	\vhEntry{0.1}{2014-05-10}{HaaTa}{Initial Draft}
	\vhEntry{0.2}{2014-05-11}{HaaTa}{Finished trigger section}
	\vhEntry{0.2a}{2014-05-15}{HaaTa}{Finished result section}
	\vhEntry{0.2b}{2014-05-17}{HaaTa}{Initial pass at USB Code Table}
	\vhEntry{0.2c}{2014-05-18}{HaaTa}{Finished USB Code Table}
	\vhEntry{0.3}{2014-09-07}{HaaTa}{Initial KLL implementation}
	\vhEntry{0.3a}{2014-11-21}{HaaTa}{Adding Defines}
	\vhEntry{0.3b}{2015-05-02}{HaaTa}{Adding Media Keys and Null Key}
	\vhEntry{0.3c}{2015-09-29}{HaaTa}{Adding soft replacement}
	\vhEntry{0.3d}{2015-10-17}{HaaTa}{Adding more consumer control codes}
	\vhEntry{0.4}{2015-05-19}{HaaTa}{Adding key state scheduling}
	\vhEntry{0.5}{2015-12-02}{HaaTa}{Adding Pixel Output Control}
	\vhEntry{0.5a}{2015-12-10}{HaaTa}{Adding array variables}
	\vhEntry{0.5b}{2016-04-20}{HaaTa}{Adding trigger macro isolation}
	\vhEntry{0.5c}{2016-10-09}{HaaTa}{Adding pixel addressing types}
	\vhEntry{0.5.4}{2018-03-24}{HaaTa}{Adding new trigger options}
	\vhEntry{0.5.5}{2018-05-03}{HaaTa}{Adding Locale support}
	\vhEntry{0.5.6}{2018-08-11}{HaaTa}{Adding rotation trigger}
	\vhEntry{0.5.7}{2019-02-03}{HaaTa}{Adding clearactive replace state}
\end{versionhistory}

%%% Tables %%%
\printtables

%%% Summary %%%
\chapter{Summary}

This document outlines a complex keymapping scheme for custom keyboard firmware.
It is based upon the concept of trigger:result pairs and ways to configure both parts of the pair (input and output).
A variety of macros are supported along with analog keyswitches.

In addition, this keymapping scheme supports static and dynamic layering as well as supporting arbitrary keyboard features that may or may not be available on a given keyboard.


%%% Scope %%%
\chapter{Scope}

Hardware keymappings for keyboards traditionally has been cumbersome.
At best, each layer can be defined as a set of predefined trigger:result pairs.
A trigger being a key and a result being a key press (e.g. USB Code).
Some keyboards allow for macros; however, these macros usually have to be custom programmed to the microcontroller (most, if not all custom keyboard firmware), or have very limited scope (Kinesis Advantage Pro and some Cherry POS keyboards such as the G80-8200).

Secondly, keymap layering (e.g. FN Layer) almost universally supports distinct layers (in the firmware that support this).
However, no current firmware support ``partial layering''.
Partial layering is only modifying a part of the current layout, similar to how XKB handles remapping Ctrl and CapsLock.

To reduce size constraints, EEPROM storage of keyboard layouts will not be considered.
It may be possible to do post compilation keymap modifications, but this is considered optional for an implementation.

Finally, as of current (2015-05-10), no keymapping mechanism supports analog keyswitches for use in typing.
This is mostly lack of availability of analog keyboard switches.


%%% Variables %%%
\chapter{Variables}

Variables serve two purposes.
The first, to give information to the compiler both for informational and controlling internal keyboard firmware features.
The second is for keymapping convenience.
In most cases variables will only be used for compiler information.

Variables used for non-compiler purposes must be integers.
The compiler will enforce this.

Variables are defined as follows.

\begin{lstlisting}
<variable name> = <contents>;
\end{lstlisting}

Each command ends with a semi colon, there are no exceptions to this.
Every keymap file will have the following variable.

\begin{lstlisting}
Name = myKeymapFile;
\end{lstlisting}

If spaces are required for either the variable name or variable contents, doubles quotes may be used.
The spaces will be internally converted to underscores if the variable is used for non-informational purposes.
Variable names can only use A-Z, a-z, 0-9 and underscores and must not start with a number.

\begin{lstlisting}
"space containing variable" = "space-ful variable";
mix = "And Match";
\end{lstlisting}

In some cases it's necessary to define an array variable to give the compiler better context of the variables.
Particularily when defining separate buffer locations that need to be combined into a single index range (e.g. Pixel Control).
Arrays can be defined by index or space separated list.
If the compiler is unsure about how to use the array it will default to a space separated list as the output (e.g. Defines).

\begin{lstlisting}
variable[0] = "index 1 info";
variable[1] = 56;
var2[] = item1 item2 "item 3";
\end{lstlisting}


 %% Defines %%
\section{Defines}
\label{sec:Defines}

Defines are a special case of variables that are used to influence static configuration options of the keyboard.
This allows information to be given to the keyboard firmware before it is compiled.
For example, defining the word size of the macro definition (i.e.\ using 8 bit instead of 32 bit when only 256 positions are required).

Defines are set the same way as variables are.
However, and additional configuration is required to indicate which variables set defines for the compiled firmware.
This help save the user from not setting critical defines that the firmware requires as warning messages are generated by the compiler when they are missing.

\begin{lstlisting}
<capability name> => <corresponding C/C++ define>;
\end{lstlisting}

Each keyboard will likely have a different set of configurable defines which will be exposed.
By default, they will have set values; however, these values can be overridden.

\begin{lstlisting}
# Define declaration
myDefine => myCDefine;

# Default value
myDefine = 23;

# Override
myDefine = 144;
\end{lstlisting}

In order to pass strings, the desired quotes must be double-quoted.

\begin{lstlisting}
myDefine => myCDefine;

# Correct
myDefine = '"This is a good string"';

# Possibly incorrect
myDefine = "This is an iffy string";
\end{lstlisting}

The second variable assignment will not pass the double quotes to the generated file so the result will not be interpreted as a C-String by the C/C++ compiler.


%%% Capabilities %%%
\chapter{Capabilities}
\label{chpt:Capabilities}

Capabilities define what the keyboard can do.
At basic level, each keyboard has the capability to send USB Codes.
Some keyboards have solenoids that can be fired, others have clickers.
Capabilities are read by the compiler to cross-reference the functions that control that capability to the designated key press.

If a capability is specified as a result to a keypress and is not defined for the specified keyboard, it is ignored and removed from the compiled keyboard mappings (for each of the layers it is defined on).

In general, capabilities specific to each keyboard are defined in the keyboard specific Scan Code to USB Code keymap.
Capabilities such as clickers and solenoids are defined using a standard so that their functionality does not have to be redefined for each keymap (i.e.\ keymaps are transferable between keyboards even with special functionality that does not apply to every keyboard).

Capabilities are defined as follows.

\begin{lstlisting}
<capability name> => <corresponding C/C++ function>;
\end{lstlisting}

The arguments of the C/C++ function must be specified, but not defined.
Capabilities are used to describe to both the compiler and user, what the special functionality is supposed to do.
The argument names themselves should be descriptive names containing no spaces, start with a letter and may use a-z, A-Z, 0-9 and underscore.
The colon after each argument specifies the byte length of the argument.
This number must be an positive integer.

Capabilities have two implied arguments, state and state type.
These arguments are set by the TriggerMacro that signaled this Capability.

\begin{lstlisting}
# Correct
myCapability => myCFunction( arg1:1, arg2:2 );

# Incorrect, defines the arg2 as 25
yourCapability => myCFunction( arg1:1, 25:2 );
\end{lstlisting}

Refer to Section~\ref{chpt:CapabilitiesTable} for a list of common capabilities.


%%% Keymapping %%%
\chapter{Keymapping}

Keymapping is the main purpose of KLL.
KLL is designed to deal with most kinds of macros and layering (in addition to keyboard specific functions).

Keymapping is defined in two parts.
The first part is call the trigger.
The trigger defines what the signal of a keypress (e.g.\ press a) is that will be used to generate a result (e.g.\ send USB Code a).
Triggers can be simple or very complex.
Defined as Scan Codes (non-portable) and USB Codes (portable).

The second part is called a result.
The result is what the keyboard firmware is designated to do once the corresponding trigger is received.
Results vary from sending single USB Codes, to USB Code Macros, to enabling keyboard specific capabilities and toggling other keymapped layers.
Defined as USB Codes and Capabilities.
If an undefined Capability is used it is ignored and the trigger:result pair is removed from the keymap.

As a note, time based key sequences (e.g.\ entering keys within a specified period of time) are not supported for both the trigger and result mechanisms.
These may be added in a future version of KLL if enough demand (and a sensible/scalable implementation is proposed).

The following is the general syntax of the trigger:result pair.

\begin{lstlisting}
<trigger> : <result>;
\end{lstlisting}

There are four variants of the trigger:result pair.
Depending on which variant is used, assignment of the result changes.

\begin{lstlisting}
<trigger> :  <result>; # Replaces results on trigger
<trigger> :: <result>; # Soft layer replacement
<trigger> :+ <result>; # Adds results to trigger
<trigger> :- <result>; # Removes result from trigger
\end{lstlisting}

In general, the normal trigger:result pair will be used to replace/change keymapping.
The trigger::result pair is a soft layer replacement which is used to only do replacement if two conditions are met: not the default layer and the trigger being replaced is not the same as defined in the base map.
This is useful for making broad replacement rules for all layers but only come into effect if actually being used.
An adding trigger:+result pair is useful when adding an extra action to a key of macro of keys (e.g.\ solenoid press).
A subtraction trigger:-result pair is useful when compiling in many layouts and there are certain additions that are undesirable.


% Isolation
\subsection{Isolation}

Sometimes it's useful to define macros that are comprised of simple macros already linked to results.
Instead of using the more sophisticated state scheduling in Section~\ref{subsubsec:ScanCodeStateScheduling}, trigger assignment isolation can be used.
For example, say the A and B keys are already assigned to output `A' and `B' respectively.
However, you want A+B together to output `Q', but not `A' or `B'.
Isolation macros, when triggered will cancel any other macros that may be active or may activate during that processing loop.
This means if you pressed A, then B, `A' then `Q' will be displayed.
And if you keep holding down B and A, then `Q' will start repeating.
If A is released, then `B' will output.
So, to just output `Q' both A and B must be pressed then released during the same scan cycle.

\begin{lstlisting}
<trigger> i:  <result>; # Replaces results on trigger
<trigger> i:: <result>; # Soft layer replacement
<trigger> i:+ <result>; # Adds results to trigger
<trigger> i:- <result>; # Removes result from trigger
\end{lstlisting}

If more than one isolation trigger is activated at the same time the behaviour is undefined and any one of the isolated triggers may take priority depending on how the firmware processes the inputs.



%% Trigger
\section{Trigger}

The trigger defines the conditions required for a specific keymapping.
These conditions can be as simple as a single key press or as complex as sequences of key combinations.

A trigger is defined for each type of source.
For a keyboard, the source types are Scan Codes and USB Codes.
Other types of source types include axis control (mouse), rotation control (mouse wheel), indicator codes (keyboard lock lights).

Beyond this, it is possible to specify ranges of sources to do the same function.
Require a sequence of sources entered to enable a function.
Use a combination of sources to trigger a capability.

For each type of source a state may be used to control what the trigger of the source is.
With analog switches it's possible to define a percentage value threshold for a key.

Precedence is evaluated as follows: Scan Code/USB Code/Source then range, then combination and finally sequence.
Ranges of combinations does not make sense, nor does combinations of sequences.


% Sources
\subsection{Sources}

A source is a control interface that generates a change in state that, using a measure or threshold that can trigger an event.
For keyboards, Scan Codes are native identifiers to the keyboard firmware.
Whereas USB Codes are the identifiers used to send over USB to the OS.
After compilation, all keymappings are mapped to Scan Codes; however, using Scan Codes in keymap files is not portable between different keyboards so their use is discouraged outside of defining the default Scan Code to USB Code keymap.

Logical mappings are not used for other types of control such as axis control, rotation control and indicator codes.


% Scan Code
\subsection{Scan Code}
\label{subsec:Scan_Code}

Scan Codes are the native identifier for keypresses on keyboard firmware.
In general, Scan Codes should not be used for defining keymaps; however, they are required for defining the initial Scan Code to USB Code mapping for each keyboard.

Each trigger must identify whether it is a Scan Code or USB Code.
For Scan Codes it is prefixed with an S.

\begin{lstlisting}
S<Scan Code> : <result>;
\end{lstlisting}

Scan Codes can be defined as either hex or decimal numbers.

\begin{lstlisting}
S0x2A : <result>;
S124  : <result>;
\end{lstlisting}

The compiler will error if it is specified in the keymap explicitly.

For keyboards that are interconnect capable, it is possible to set the interconnect index for that keyboard.
By default, a scan code is defined as index 0.
This can be changed using the \textbf{ConnectId} variable.
This variable is effective immediately.

\begin{lstlisting}
S0x2A : <result>; # Defaults to index 0
ConnectId = 2;
S0x2A : <result>; # Defaults to index 2
S124  : <result>; # Also defaults to index 2
\end{lstlisting}


% State Scheduling
\subsubsection{State Scheduling}
\label{subsubsec:ScanCodeStateScheduling}

Instead of processing the trigger immediately, as done by default, it is also possible to put additional state conditions on the keypress.
Keyswitch trigger events have 6 different states that can be specified: press (\textbf{P}), hold (\textbf{H}), release (\textbf{R}), off (\textbf{O}), unique press (\textbf{UP}) and unique release (\textbf{UR}).
State conditions are defined using parenthesis after the keyswitch definition.

The press event is the default schedule event for keyswitches.
The release event is useful for signalling an event after the key has been released rather than when it was pressed.
Unique press is a special state that requires no other keyswitch is currently in the press or hold state.
Unique release is another special state requiring no additional keypresses since this key was pressed.
This is often referred to as ``tap'' keys.

\begin{lstlisting}
S100     : <result>;
S100(P)  : <result>; # Same as above
S101(UP) : <result>; # Tap key
S102(UR) : <result>; # Single key input
S103(R)  : <result>; # Single-shot on release
\end{lstlisting}

Hold and off are non-triggering events so they cannot be used to trigger results alone.
This means they \textbf{must} be part of a combination with a triggering event (see Section~\ref{subsec:Combination}).

\begin{lstlisting}
S104(H)           : <result>; # Incorrect
S104(H) + S105(O) : <result>; # Incorrect
S100 + S104(H)    : <result>; # Correct, holding a key
S101 + S105(O)    : <result>; # Correct, key is not pressed
\end{lstlisting}


% Key Positioning
\subsubsection{Key Positioning}
\label{subsubsec:keypositioning}

It is possible to give a physical location to a particular Scan Code.
This allows for configuration utilities to automatically generate the layout as well as give hints to pixel positioning (i.e. backlight keys).
Key positioning is generally completely optional and needs to be done once for every physical layout configuration of the device.

Each key may be assigned the following options, by default they are set to 0.

\begin{itemize}
\item x, y, z positions (mm)
\item rx, ry, rz rotations (deg)
\end{itemize}

The units are in mm and degrees respectively.

\begin{lstlisting}
# Set the x position 20 mm and x rotation 15 deg
S120 <= x:20,rx:15;
\end{lstlisting}

Currently keycap shape is not supported.
Though suggestions are welcome on how best to support them (especially the odd shaped ones).


% USB Code
\subsection{USB Code}
\label{subsec:USB_Code}

USB Codes define how USB understands keyboard output press/release events and are defined by the USB HID Spec.
USB Codes are the recommend identifier when defining triggers for KLL keymaps.

Each trigger must identify whether it is a Scan Code or USB Code.
For USB Codes it is prefixed with a U.

\begin{lstlisting}
U<USB Code> : <result>;
\end{lstlisting}

USB Codes can be defined in a number of ways.
Like Scan Codes, the USB Code can be defined directly using hex or decimal numbers (not recommended).
The recommended way is to use the descriptive names for the USB Codes.
Descriptive names are always strings, and must be enclosed with double quotes.

\begin{lstlisting}
U0x2A : <result>;
U124  : <result>;
U"A"  : <result>;
U"a"  : <result>; # Same as previous
\end{lstlisting}

Refer to Section~\ref{chpt:USBCodeTable} for the complete list of standard (US ANSI) USB Codes.
For dealing with non-ANSI keyboard locales and layouts please refer to Section~\ref{sec:locales}.

There is an important difference when using a USB Code as the the Trigger as compared to a Scan Code.
When an assignment is done using a USB Code, a lookup must occur to find the original Scan Code that the USB Code was assigned to.
Since assignment order matters this can become confusing.

\begin{lstlisting}
U"s" : U"r";
U"r" : U"p";
\end{lstlisting}

This has an unintended side-effect.
All R's will be assigned to P.
If the user is aware, the problem isn't too bad to deal with.
However, this is a very tricky problem to deal with for GUI keymap programs as it is not intuitive to enforce order.

Therefore, to resolve this issue, all assignments using a USB Code trigger must be cached until the current kll file has finished parsing.
Then the cached assignments can be applied to the current layer.
This will allow mass assignment of keys without having to worry about the order of assignment.
If re-assignment is needed, use another kll file or a Scan Code trigger.


% State Scheduling
\subsubsection{State Scheduling}
\label{subsubsec:trigusbstateschedule}

All of state schedulers used for Scan Code triggers are applicable for USB Code triggers (see Section~\ref{subsubsec:ScanCodeStateScheduling}).

\begin{lstlisting}
U"A"           : <result>;
U"A"(P)        : <result>; # Same as above
U"B"(UP)       : <result>; # Tap key
U"C"(UR)       : <result>; # Single key input
U"D"(R)        : <result>; # Single-shot on release
U"A" + U"Q"(H) : <result>; # Correct, holding a key
U"B" + U"W"(O) : <result>; # Correct, key is not pressed
\end{lstlisting}


% Button Code
%\subsubsection{Button Code}
% TODO
% B[<num>]


% Linear Code
%\subsubsection{Linear Code}
% TODO - Use +/- for relative state
% L[<num>]


% Rotation Code
%\subsubsection{Rotation Code}
% TODO - Use +/- for relative state
% R[<num>]


% Analog
\subsubsection{Analog States}

For analog keyboard switches, it is useful to specify a trigger at a press percentage as press/release behaviour needs to be defined.
The expected implementation defines a press when the key stops moving or begins to move back up.
If a stopped key moves further down (past the next trigger), stops or begins to move back up the next trigger is used and the previous on is released.
Again, if the stopped key moves back up going back over a past trigger, stops (or with a slight delay before going back up further) this past trigger will activate, releasing the previous trigger.

Analog triggers are specified using parenthesis.

\begin{lstlisting}
<Code>(<Analog Value>) : <result>;
\end{lstlisting}

Both USB Codes and Scan Codes can be used with analog keyboards signals.
Mapping analog triggers gets more tricky, so it is recommended that they are only used with USB Codes.
Analog triggers are specified from 0 to 100 using integers only.
0 is a special case and is only pulsed when it is reached rather than held.

\begin{lstlisting}
S0x2A(10) : <result>; # 10% press
S0x2A(80) : <result>; # 80% press
U124 (50) : <result>; # 50% press
U"A"  (0) : <result>; #  0% pulse
U"a" (52) : <result>; # 52% press
\end{lstlisting}

If the keyboard is not analog, any trigger with an analog trigger will be ignored.
And inversely, if the Scan Code/USB Code maps to an analog key, a normal press/release trigger will be ignored.


% LED Indicator Code
\subsection{LED Indicator Code}
\label{subsec:LED_Indicator_Code}

Instead of using a keypress as the trigger to a macro, it is also possible to use the state of a HID LED Indicator such as Caps Lock.

\begin{lstlisting}
I<Indicator Code> : <result>;
\end{lstlisting}

Indicator codes can be specified using a numeric or string identifier.

\begin{lstlisting}
U"a" + I2         : <result>; # a + CapsLock
U"a" + I0x3       : <result>; # a + ScrollLock
U"a" + I"NumLock" : <result>; # a + NumLock
\end{lstlisting}

Refer to Section~\ref{chpt:LEDIndicatorCodeTable} for the complete list of LED Indicator Codes.


% State Scheduling
\subsubsection{State Scheduling}

Similar to keypresses, indicator codes have four states: activate (\textbf{A}), on (\textbf{On}), deactivate (\textbf{D}) and off (\textbf{Off}).
Also similar to keypresses, on and off cannot be used to trigger a macro.
While activate and deactivate can be used as a trigger to a macro.
The default state for LED Indicator codes is activate.

\begin{lstlisting}
I"NumLock"    : <result>;
I"NumLock"(A) : <result>; # Same as above
I"NumLock"(D) : <result>; # NumLock disable
\end{lstlisting}

The on and off states must be specified with another trigger as they are non-trigger conditions.

\begin{lstlisting}
I"NumLock"(On) + U"c"  : <result>; # Correct
U"a" + I"NumLock"(Off) : <result>; # Correct
I"NumLock"(Off)        : <result>; # Incorrect
I"NumLock"(On), U"d"   : <result>; # Incorrect
\end{lstlisting}

Keep in mind that the LED Indicator Codes are controlled by the host OS and not by the keyboard.
There is always the chance that the OS tries to do something strange with the LEDs that may not match what the OS itself is doing (i.e. CapsLock led is off, but all keys are outputing as caps).
In this case there is really nothing the keyboard can do to right itself.


% Layer State
\subsection{Layer State}
\label{subsec:Layer_State}

Layer state may be used as a trigger to a macro.
In most cases using \textbf{Layer} is preferrable as it will trigger using any of: Shift, Latch and Lock.
While using the specific triggers is more restrictive.

\begin{lstlisting}
Layer[1]      : <result>; # Layer 1
LayerShift[2] : <result>; # Layer 2 shift
LayerLatch[2] : <result>; # Layer 2 latch
LayerLock[2]  : <result>; # Layer 2 lock
\end{lstlisting}


% State Scheduling
\subsubsection{State Scheduling}

Similar to keypresses, layers have four states: activate (\textbf{A}), on (\textbf{On}), deactivate (\textbf{D}) and off (\textbf{Off}).
Also similar to keypresses, on and off cannot be used to trigger a macro.
While activate and deactivate can be used as a trigger to a macro.
The default state for a layer is activate.

\begin{lstlisting}
Layer[1]    : <result>;
Layer[1](A) : <result>; # Same as above
Layer[1](D) : <result>; # Layer 1 disable
\end{lstlisting}

As with indicators (Section~\ref{subsec:LED_Indicator_Code}, the on and off states must be specified with another trigger as they are non-trigger conditions.

\begin{lstlisting}
Layer[1](On) + U"c"  : <result>; # Correct
U"a" + Layer[1](Off) : <result>; # Correct
Layer[1](Off)        : <result>; # Incorrect
Layer[1](On), U"d"   : <result>; # Incorrect
\end{lstlisting}


% Animation
\subsection{Animation}

Refer to Section~\ref{sec:Animation_Triggers} on Animation Trigger usage.


% Generic Triggers
\subsection{Generic Triggers}

For custom defined triggers there may not exist KLL syntax yet.
In this case, generic triggers can be used to define the trigger identifier and code directly.
Generally these trigger identifiers shouldn't change; however, don't consider them portable between major firmware versions.
But an effort is made not to change the identifier numbering.

The code is always specified as an 8 bit number.
For codes higher than 255 you'll need to determine the identifier associated with that code.
For example, \textbf{0} is Switch Bank 1 corresponding from 0 to 255 and \textbf{2} is Switch Bank 3 corresponding from 512 to 767.

While it is possible to trigger on the Animation Bank (See Section~\ref{subsubsec:Trigger_Identifiers}) these may be dynamically indexed based on which animations are defined.
So you will almost certainly run into issues using a Generic Trigger.

Generic triggers do not support ranges as seen in Section~\ref{subsection:Range}.

\begin{lstlisting}
S2     : <result>;
T[0,2] : <result>; # Same as S2
S257   : <result>;
T[1,1] : <result>; # Same as S257

I"NumLock" : <result>;
T[4,0]     : <result>; # I"NumLock"

Layer[1] : <result>;
T[9,1]   : <result>; # Layer[1]

# On host sleep trigger
T[17,0] : <result>;
# 1 minute after host sleep trigger
T[17,1] : <result>;

# After device inactivity
T[19,0] : <result>;
# 2 minutes after device usages stops
T[19,2] : <result>;

# Two rotation triggers, 0 and 1
T[20,0](0) : <resulta1>;
T[20,0](1) : <resulta2>;
T[20,0](2) : <resulta3>;
T[20,1](0) : <resultb1>;
T[20,1](1) : <resultb2>;

# Rotate right rotation 0
<trigger> : rotate(0, 1);

# Rotate left rotation 1
<trigger> : rotate(1, -1);
\end{lstlisting}


% List of Trigger Identifiers
\subsubsection{Trigger Identifiers}
\label{subsubsec:Trigger_Identifiers}

The current list of trigger identifiers and their code ranges.

\begin{ltable}{triggeridentifiers}{ l | r | l }{Table of Trigger Identifiers and Code Ranges}

\textbf{Identifier Value} & \textbf{Name} & \textbf{Code Offset} \\
\hline
\hline

% List of Trigger Identifiers

0x00 (0) & Switch Bank 1 & 0-255 \\
0x01 (1) & Switch Bank 2 & 256-511 \\
0x02 (2) & Switch Bank 3 & 512-767 \\
0x03 (3) & Switch Bank 4 & 768-1023 \\
0x04 (4) & LED Bank 1 & 0-255 \\
0x05 (5) & Analog Bank 1 & 0-255 \\
0x06 (6) & Analog Bank 2 & 256-511 \\
0x07 (7) & Analog Bank 3 & 512-767 \\
0x08 (8) & Analog Bank 4 & 768-1023 \\
0x09 (9) & Layer Bank 1 & 0-255 \\
0x0A (10) & Layer Bank 2 & 256-511 \\
0x0B (11) & Layer Bank 3 & 512-767 \\
0x0C (12) & Layer Bank 4 & 768-1023 \\
0x0D (13) & Animation Bank 1 & 0-255 \\
0x0E (14) & Animation Bank 2 & 256-511 \\
0x0F (15) & Animation Bank 3 & 512-767 \\
0x10 (16) & Animation Bank 4 & 768-1023 \\
0x11 (17) & Sleep Bank 1 & 0-255 \\
0x12 (18) & Resume Bank 1 & 0-255 \\
0x13 (19) & Inactive Bank 1 & 0-255 \\
0x14 (20) & Active Bank 1 & 0-255 \\
\hline
% 0x15 - 0xFE Reserved
0xFF (255) & Debug State & \\
 % See .tex file for data

\end{ltable}


% Range
\subsection{Range}
\label{subsection:Range}

For convenience, it is also possible to define a range of triggers rather than explicitly defining each of the trigger:result pairs.
This is useful for defining things such as using the clicker speaker for every single key or just the letter keys.

Ranges are defined as the numerical range of either USB Codes or Scan Codes (use not recommended).
Fortunately this means ranges such as A-Z will work has USB HID defines the USB Codes in order.
Unfortunately the range 0-9 will not work as the USB Codes are organized: 1, 2, 3, 4, 5, 6, 7, 8, 9, 0 just like on a keyboard.
To assist with this problem, the compiler will warn when this situation occurs.
Forward and reverse ranges are possible.
There is no wrap-around.
Scan Codes and USB Codes cannot be mixed.

To define a range, two additional pieces of syntax are required: square brackets and range specifier.

\begin{lstlisting}
S[0x2A-0x50] : <result>;
S[12-43]     : <result>;
S[69]        : <result>; # Also a valid definition
U[124-43]    : <result>; # Reverse range
U["A"-"Z"]   : <result>;
U["0"-"9"]   : <result>; # Compiler will warn, only 0 and 9
U["1"-"0"]   : <result>; # Correct definition of 0-9
\end{lstlisting}

In addition to a specific range, multiple single keys or ranges may also be specified.

\begin{lstlisting}
S[0x2A-0x50, 0x5]   : <result>;
S[12, 43]           : <result>;
U["A"-"Z", "Enter"] : <result>;
\end{lstlisting}

When using an analog supporting keyboard, a range is specified to that specific range.
Or the analog trigger can be specified for the entire range.

\begin{lstlisting}
S[0x2A-0x50(20), 0x5(56)] : <result>;
S[12, 43](79)             : <result>;
U["A"-"Z"(42),"Tab"](30)  : <result>; # 30, unless specified
\end{lstlisting}


% Sequence
\subsection{Sequence}
\label{subsec:Sequence}

Sequences are a type of macro.
A sequence is a set of expected inputs that must happen in order with no extra inputs of non-defined inputs.
If a non-defined input is found, the sequence is reset and waits for the first input in the sequence again.
Sequences cannot mix Scan Codes and USB Codes.

Trigger inputs in the sequence are separated by commas.
The Scan Codes and USB Codes are evaluated/expanded first, then the sequences are evaluated (Codes have higher precedence).

\begin{lstlisting}
S[0x2A-0x50], S[12], U[5]           : <result>;
U["Q"-"Y"], U["Enter"]              : <result>;
U["A"-"Z"(42),"Tab"](30), U[12](53) : <result>;
\end{lstlisting}

Analog triggers cannot be set across multiple inputs in the sequence.

For convenience, a second sequence syntax is available only for specifying ASCII characters.
This allows for sentences to be inputted rather than the individual USB Codes.
The key difference is the use of single quotes.
If a shift is required for the specific key, it will be added as a combination (see Section~\ref{subsec:Combination}).

\begin{lstlisting}
'abc'                       : <result>;
U'T'                        : <result>;
U['Can you type this?'](23) : <result>;
\end{lstlisting}

Single quotes are only used for sentence expansion, they cannot be used for strings.


% Combination
\subsection{Combination}
\label{subsec:Combination}

Combinations are a type of macro.
A combination is a set of inputs pressed at the same time.
For example: Ctrl+Alt+Delete.
In order for a combination trigger to be signalled all of the keys must be pressed.
The combination will not end if a key is missing from the combination (as the user may not have pressed it yet).
Extra keys not part of the combination also do not cancel the combination.
Combinations cannot mix Scan Codes and USB Codes.

Trigger inputs in the combination are separated by pluses.
The Scan Codes and USB Codes are evaluated/expanded first, then sequences are evaluated and finally combinations (Codes have the highest precedence, then sequences, then combinations).

\begin{lstlisting}
U["Q"-"Y"] + U["Enter"]              : <result>;
S[0x2A-0x50] + S[12], U[5]           : <result>;
U["A"-"Z"(42),"Tab"](30) + U[12](53) : <result>;
'Can you type this?'(23) + 'a'(43)   : <result>;
\end{lstlisting}

Analog triggers cannot be set across multiple inputs in the combination.


% Timing
\subsection{Timing}
\label{subsec:trigtiming}

It is possible to assign a time to a given trigger.
This allows for sophisticated time based input sequences.

The granularity of the fundamental time base depends on the hardware.
This time base is advertised by the underlying hardware to the KLL compiler using the \textbf{timeBase} define.

The following are the accepted time units.
\begin{itemize}
	\item \textbf{s} - Seconds
	\item \textbf{ms} - Milliseconds
	\item \textbf{us} - Microseconds
\end{itemize}

Fractional numbers are also valid (e.g. 1.43~s).
If the number will be rounded to the nearest fundamental time unit.
Avoid using the fundamental time unit directly as it will not behave the same way between different keyboards.

The format is as follows.

\begin{lstlisting}
<identifier>([<state scheduler>][:<time>]) : <result>;
\end{lstlisting}

Hold A for at least 300~ms.

\begin{lstlisting}
U"a"(300ms)   : <result>;
U"a"(H:300ms) : <result>; # Same
\end{lstlisting}

Hold A for at most 300~ms.

\begin{lstlisting}
U"a"(R:300ms) : <result>;
\end{lstlisting}

More complicated state scheduling is also possible.
Press timers may be used to have combos that require a key pressed first then after a certain amount of time another key must be pressed, which then completes the element in the sequence.

\begin{lstlisting}
U"a" + U"b"(P:1s) : <result>; # Press B after 1s
U"a" + U"b"(H:1s) : <result>; # B must be held for at least 1s
U"a" + U"b"(R:1s) : <result>; # B cannot be held for more than 1s
\end{lstlisting}

Non-trigger events cannot be used with a time specifier.
This includes: \textbf{On}, \textbf{Off} and \textbf{Hold}.
Hold can only be used when using multiple time triggers.

Multiple time triggers may be used to define more complex input sequences.
For example, to press A, require a wait time of 50~ms then hold for at least 100~ms, but no longer than 200~ms.

\begin{lstlisting}
U"a" + U"b"(P:50ms,H:100ms,R:200ms);
\end{lstlisting}


%% Result
\section{Result}

The result is the action, or set of actions, after the conditions for the trigger have been satisfied.
The result can be as simple as a single USB Code output or as complex as signalling a keyboard specific clicker and outputting a combinational macro.
A Scan Code cannot be used as a result.

Like triggers, there are two basic types of results: keyboard specific functionality and USB Codes.
Both of these results are referred to as capabilities.
A basic capability of every keyboard is to send USB Codes, while other keyboards may have other capabilities such FN layers or clickers.
When a trigger is satisfied each of the results assigned to this trigger are signalled.

Much of the syntax is the same as with triggers.
The major difference being, no analog qualifiers and no ranges.
Some state scheduling qualifiers also may not apply in all situations, if at all.


% USB Code
\subsection{USB Code}

USB Codes are the primary use of the result part of the trigger:result pairs.
Result USB Codes are defined the same way as trigger USB Codes (Section~\ref{subsec:USB_Code}).

\begin{lstlisting}
<trigger> : U0x2A;
<trigger> : U124;
<trigger> : U"A";
<trigger> : U"a"; # Same as previous
\end{lstlisting}

Refer to Section~\ref{chpt:USBCodeTable} for the complete list of standard (US ANSI) USB Codes.
For dealing with non-ANSI keyboard locales and layouts please refer to Section~\ref{sec:locales}.


% State Scheduling
\subsubsection{State Scheduling}
\label{subsubsec:resultusbcodestateschedule}

It is also possible to schedule output sequences just as with USB Code triggers (see Section~\ref{subsubsec:trigusbstateschedule}).
Result USB Codes have three scheduling types: press (\textbf{P}), hold (\textbf{H}) and release (\textbf{R}).
Off, unique press and unique release have no meaning for output and therefore are not available.
The hold state can only be used with a timing specifier (see Section~\ref{subsec:resulttiming}).

By default, the press specifier is used and the key is released before the next sequence element and/or the end of the sequence.

\begin{lstlisting}
<trigger> : U"A";
<trigger> : U"A"(P,R); # Same as previous
<trigger> : U"A"(P), U"A"(R);
<trigger> : U"A"(P); # Does not release a
<trigger> : U"A"(R); # Only releases a if active/pressed
\end{lstlisting}


% None
\subsection{None}

Sometimes it is useful to block the fall-through to a previous layer and have the key do nothing.
The \textbf{None} keyword is case sensitive.

\begin{lstlisting}
<trigger> : None;
\end{lstlisting}


% Consumer Control Code
\subsection{Consumer Control Code}

Consumer Control Codes, or Media Keys are a USB HID control type that allows you to control media functions.
These include global hotkeys such as Play, Next and Previous.

Similar to USB Codes (Section~\ref{subsec:USB_Code}), Consumer Control Codes can be defined as numbers or by its symbolic name.
It is important to note that there is no support for sequences, combinations or ranges of Consumer Control Codes.

\begin{lstlisting}
<trigger> : CON0xB0;
<trigger> : CON176;
<trigger> : CON"Play";
<trigger> : CON"play"; # Same as previous
\end{lstlisting}

Refer to Section~\ref{chpt:ConsCodeTable} for the complete list of Consumer Control Codes.

Consumer Control Codes may also use state scheduling. See Section~\ref{subsubsec:resultusbcodestateschedule} for usage details.


% System Control Code
\subsection{System Control Code}

System Control Codes are a USB HID control type that allows you to system level functions.
These include global hotkeys such as Power, Sleep and Eject.

Similar to USB Codes (Section~\ref{subsec:USB_Code}), System Control Codes can be defined as numbers or by its symbolic name.
It is important to note that there is no support for sequences, combinations or ranges of System Control Codes.

\begin{lstlisting}
<trigger> : SYS0x82;
<trigger> : SYS130;
<trigger> : SYS"Sleep";
<trigger> : SYS"sleep"; # Same as previous
\end{lstlisting}

Refer to Section~\ref{chpt:SysCodeTable} for the complete list of System Control Codes.

System Control Codes may also use state scheduling. See Section~\ref{subsubsec:resultusbcodestateschedule} for usage details.


% Sequence
\subsection{Sequence}

A result sequence is a series of outputted USB Codes which are split between each USB output buffer refresh.
This is similar to a trigger input sequence.
The syntax is the same as trigger sequences (Section~\ref{subsec:Sequence}.

\begin{lstlisting}
<trigger> : U["Q"-"Y"], U["Enter"];
<trigger> : U["A"-"Z","Tab"], U[12];
\end{lstlisting}

Single quoted strings will be literally evaluated.
Any characters requiring a Shift key will have it added as a combination as part of the result.

\begin{lstlisting}
<trigger> : 'abc';
<trigger> : U'T';
<trigger> : U['Can you type this?'];
\end{lstlisting}


% Combination
\subsection{Combination}

A result combination is a set of keys sent out during the same USB output buffer.
The syntax is the same as trigger combinations (Section~\ref{subsec:Combination}.
Keep in mind, if the keyboard does not support NKRO then some result combinations will not be possible.

\begin{lstlisting}
<trigger> : U["Q"-"Y"] + U["Enter"];
<trigger> : U["A"-"Z","Tab"] + U[12];
<trigger> : 'Can you type this?' + 'a';
\end{lstlisting}


% Capability
\subsection{Capability}

In addition to outputting USB Codes, capabilities can be triggered.
Technically, outputting USB Codes is a capability of the keyboard, but it is always considered to exist.
If a capability is specified as a result, but is not available for that keyboard, the trigger:result pair is ignored by the compiler and removed from the keymap.

To specify a capability as the result, define the capability name and any arguments required for the capability.
To use the default parameter for an argument, set the argument as NULL.

\begin{lstlisting}
# Defined capability
myCapability => myCFunction( arg1:1, arg2:1 );

# Using default first argument, and 25 for the second
<trigger> : myCapability( NULL, 25 );
\end{lstlisting}

To specify one of the arguments as the analog/press (press, hold, release) variable from the trigger, specify the argument as "output".

\begin{lstlisting}
# Defined capability
myCapability => myCFunction( arg1:1, arg2:1 );

# output specifies the analog/press result
<trigger> : myCapability( NULL, output );
\end{lstlisting}

Refer to Section~\ref{chpt:CapabilitiesTable} for a list of common capabilities.


% Timing
\subsection{Timing}
\label{subsec:resulttiming}

Similar to triggers, timing may also be used for sophisticated output sequences.
See Section~\ref{subsec:trigtiming} for detailed background on timing units and fundamental time steps.

The format is as follows.
\begin{lstlisting}
<trigger> : <identifier>([<state scheduler][:][<time>]);
\end{lstlisting}

To press A for 300~ms.

\begin{lstlisting}
<trigger> : U"a"(300ms);
<trigger> : U"a"(P,H:300ms,R); # Same as above
\end{lstlisting}

To press A after 20~ms, hold for 300~ms, then release 5~ms later.

\begin{lstlisting}
<trigger> : U"a"(P:20ms,H:300ms,R:5ms);
\end{lstlisting}

Implementations will most likely not support result macro stacking/queuing.
If the timing is too long, any more triggers of the result macro will likely be ignored until the original trigger result completes.

It is also possible to schedule capabilities as well.
While capabilities will accept different states (Press, Hold, Release) the capability may ignore that specification depending on how it was implemented.
The only part guaranteed is that at each scheduled event the capability will be called with the given state information.
If a state is scheduled and the trigger input has an analog value (or other trigger specific value) associated with it that value will be discarded.
This means that the output will no longer be analog and be handled as a toggle mechanism.

\begin{lstlisting}
# Defined capability
myCapability => myCFunction( arg1:1, arg2:1 );

# output specifies the analog/press result
# Analog context is discarded
<trigger> : myCapability( NULL, output )(P, H:300ms, R);
<trigger> : myCapability( NULL, output )(100ms);
\end{lstlisting}


%% Locales %%
\section{Locales}
\label{sec:locales}

When the USB HID Spec was designed, the main assumption was that all the naming was to be done for US (ANSI) keyboards first.
Then, let the OS decide whether or not to convert each keypress to a different locale, such as French AZERTY.
For KLL, this poses two challenges.

The first challenge is that the OS must be configured to use a particular layout, something which only the user can truly determine.
Additional software can help to deduce what the OS configured layout is, but it requires OS specific code to query this information.
And, to make matters worse, it's possible for users to define custom layouts that the user created.

The second challenge is that when defining layouts in KLL it's not sufficient to just use the default ANSI labels when the user is more familiar with another locale (DE QWERTZ for example).
Since the user doesn't know the ANSI positions, a great deal of mental acrobatics (or lookup tables) are necessary to determine which codes to actually use.
This means the resulting KLL files look incorrect (and are incorrect, if the user sets the host layout to ANSI), but are correct when using the host layout they were meant for.

USB HID has only one field to help the OS to determine (and the OS is not required to use it) which host layout to use, \textbf{bCountryCode}.
Unfortunately, there are only 34 locales (35 including Undefined/Not Supported) by the USB HID Spec.
For convenience, refer to Section~\ref{chpt:LocaleTable} for a list of all the numerical USB HID layout codes.

To get around these issues, KLL uses the \href{https://github.com/hid-io/layouts}{HID-IO layouts repository}(https://github.com/hid-io/layouts) to handle all standard and non-standard USB HID layout codes.
When defining your layout, each file may set the \textbf{HIDMapping} variable which will telll the KLL compiler how to interpret the USB names, and lookup according to that locale.
If you specify multiple files, each with a different locale, each will be processed in their own locale.
However, the final locale corresponds to the last file per layer.
The final locale is only used for determining the USB HID \textbf{bCountryCode}, nothing more.
If \textbf{HIDMapping} is not set, it inherits the previous files locale.

\textbf{Please note}: It's still possible to define your locale's layout in hardware; however, many symbols are not possible using a US ANSI layout.
This also applies to things such as Programmer's Dvorak, which uses shifted characters differently than Dvorak (and US ANSI).
You are nearly always better off setting the keyboard to use a locale instead of a hardware layout, even if a hardware layout does seem more convenient.

\begin{lstlisting}
# Base KLL file used in locale tests
Name = base.locale-test;
Version = 0.1;
Author = "HaaTa (Jacob Alexander) 2018";
KLL = 0.5;
HIDMapping = default;

# Modified Date`
Date = 2018-06-04;

# Adapted from an early Kira mapping
S39 : U"Q";
S40 : U"W";
S41 : U"E";
S42 : U"R";
S43 : U"T";
S44 : U"Y";
S45 : U"U";
S46 : U"I";
S47 : U"O";
S48 : U"P";
S57 : U"A";
S58 : U"S";
S59 : U"D";
S60 : U"F";
S61 : U"G";
S62 : U"H";
S63 : U"J";
S64 : U"K";
S65 : U"L";
S73 : U"Z";
S74 : U"X";
S75 : U"C";
S76 : U"V";
S77 : U"B";
S78 : U"N";
S79 : U"M";
\end{lstlisting}

\begin{lstlisting}
# Test Layout Using a de_DE locale, above layout as base
Name = de_DE-Test;
Version = 0.1;
Author = "HaaTa (Jacob Alexander) 2018";
KLL = 0.5;
HIDMapping = de_DE;

# Modified Date
Date = 2018-06-04;

# Remap Y to Z and Z to Y
# Then check that:
# S44 -> 0x1D (US Z, DE Y)
# S73 -> 0x1C (US Y, DE Z)
# Is valid
U"Y" : U"Z";
U"Z" : U"Y";
\end{lstlisting}




%%% File Format %%%
\chapter{File Format}

The Keyboard Layout Language or KLL uses a series of plain text, readable files to define configurations.
This does not preclude GUI keymapping tools as it can generate KLL files first.

Each file represents a single keymapping layer and it is up the firmware configuration to define the precedence of each layer.
KLL uses the concept of "Capabilities" to define potential results of a keypress (Section~\ref{chpt:Capabilities}).
At a basic level these are the keycodes (USB Codes) sent out to USB, but can be things like a clicker speaker or special LED signals.


%% Comments
\section{Comments}

End-of-line comments are defined using a '\#'.
For example,

\begin{lstlisting}
# This is a comment
Not a comment # But this is a comment
\end{lstlisting}

defines the two supported styles of comments.
Block and inline comments are not supported.


%% Required Variables
\section{Required Variables}

Each keymap file has a set of required variables that must be defined.

\begin{lstlisting}
# Name of the keymap
Name = myKeymap; # Spaces will be replaced with _'s

# Version of the keymap
Version = 1.32a-HaaTa;

# Modified Date
Date = 2014-05-10;

# Author
# Multiple authors should be comma separated with the year
#  range of their contributions
# Quotes can be used if spaces are required.
Author = "HaaTa (Jacob Alexander) 2014";

# KLL Version, used to detect future incompatibilities
KLL = 0.1;
\end{lstlisting}

The \textbf{Name} and \textbf{KLL} variables are used by the KLL compiler.
Name prefixed to all keymap specific capabilities (e.g.\ myKeymap\_Latch).
KLL is for version checking to deal with future potential issues.

The rest of the variables are just informational.
But are to be propagated to combined keymaps.


%% Layer Control
\section{Layer Control}

While up to the compiler how layers are controlled, these are some guidelines on how to control layers.
Each layer must support three different activation types.

\begin{itemize}
	\item Shift - Enables keymap while held
	\item Latch - Enables keymap until the next key is pressed
	\item Lock - Enables keymap until Lock is pressed again
\end{itemize}

The recommended syntax for activating layers.
In earlier versions of KLL, capabilities were used to directly manipulate layers.
While this still works, capability syntax is a bit limiting as it does not work for corresponding Triggers.

\begin{lstlisting}
<trigger> : Layer[1];      # Shift Layer 1
<trigger> : LayerShift[2]; # Shift Layer 2
<trigger> : LayerLock[3];  # Lock Layer 3
<trigger> : LayerLatch[4]; # Latch Layer 4
\end{lstlisting}


%% Example Keymap
\section{Example Keymap}

The following is a QWERTY/US ANSI to Colemak keymap.
Only USB Codes are used as this is not a Scan Map.

\begin{lstlisting}
Name = colemak;
Version = 0.1;
Author = "HaaTa (Jacob Alexander) 2014";
KLL = 0.2a;

# Modified Date
Date = 2014-05-17;

# Top Row
'e' : 'f';
'r' : 'p';
't' : 'g';
'y' : 'j';
'u' : 'l';
'i' : 'u';
'o' : 'y';
'p' : ';';

# Middle Row
's' : 'r';
'd' : 's';
'f' : 't';
'g' : 'd';
'j' : 'n';
'k' : 'e';
'l' : 'i';
';' : 'o';

# Bottom Row
'n' : 'k';
\end{lstlisting}

When remapping ASCII characters, using single quotes is the simplest way to map keys.
Remember, all other keys must be remapped using either the USB Code or USB Code identifier.

\begin{lstlisting}
Name = capslock2ctrl;
Version = 0.1;
Author = "HaaTa (Jacob Alexander) 2014";
KLL = 0.2a;

# Modified Date
Date = 2014-05-17;

U["CapsLock"] : U["Ctrl"];
\end{lstlisting}


%%% Layering %%%
\chapter{Layering}

While the KLL files have little to do with the keymap layering implementation, there are a few features that do help with the implementation.
In general, keymaps are defined by USB codes and not Scan Codes as Scan Codes are keyboard specific.
However, for each keyboard an initial mapping must be defined from Scan Codes to USB Codes.

After compilation, each keymap is reduced down to Scan Codes as they are the unambiguous internal key indications that keyboard firmwares understand.


%% Scan Map
\section{Scan Map}

The Scan Map is the keymapping used to define each type of keyboard.
This keymapping must used Scan Codes to USB Codes, otherwise later compilation process will fail (as it won't be possible to resolve the Scan Codes).

If possible, each Scan Map should define a US ANSI-like keymap.
Any keys that are not defined by the general US ANSI standard should be defined to another reasonable USB Code.

The scan map should also define any capabilities that are provided by the keyboard.
For example, a clicker speaker.

\begin{lstlisting}
# Clicker speaker with "sound select" and volume args
myControlableClicker => clickerCFunction(sound:2, volume:1);

# Use the capability when Scan Code 0x2C is pressed
# NULL specifies the default option of the capability
S[0x2C] : myControlableClicker(NULL, 25);
\end{lstlisting}

When defining capabilities like solenoids and clickers, follow the naming convention and same number of arguments.
If there are extra capabilities for this keyboard (e.g.\ click volume), add an additional capability to control this functionality.

Refer to Section~\ref{chpt:CapabilitiesTable} for a list of common capabilities.


%% Combined Map
\section{Combined Map}

The Combined Map is a set of keymaps that have been folded (or compiled) down to a single keymap.
For example, if there is the default Scan Map and a Colemak keymap is specified, both of these keymaps will be compiled down to Scan Code triggers and the USB code results (as well as any other specified capabilities).
For keys that were not specified in the secondary maps (the highest layered map takes precedence) it will use the lower layer keymapping to be included into the Combined Map.


%% Partial Map
\section{Partial Map}

A Partial Map is a keymap that may not define every possible Scan Code.
Some uses for Partial Maps are to swap Caps Lock and Control, or define Colemak, where not all of the keys change from QWERTY/US ANSI.
Partial Maps can be used to generate a Combined Map or be compiled individually to be layered on the keyboard firmware dynamically for FN layers.


%%% Pixel Output Control %%%
\chapter{Pixel Output Control}

Some keyboards have large arrays of pixel devices such as RGB backlighting or LCD displays.
Pixel output control defines ways to specify animations and events directly using KLL without having to program custom capabilities.

Instead of triggering individual pixels, events are characterized as animation groups.
These animation groups can both a trigger and a result in order to link together animation groups with complex input sequences.

As with Scan Codes/USB Codes individual pixels also need a common way to map actions to them.
However, unlike Scan Codes, keyboards may have multiple disjunct sets of numbering schemes with different bit widths for controlling each pixel.
To get around this a pixel mapping is required in order to teach the KLL compiler where the pixels are.


%% Pixel Mapping
\section{Pixel Mapping}

Individual pixels such as LEDs or LCD pixels are not defined uniformly in hardware, especially if multiple types of hardware drivers are used in a single input device.
To get around this logical pixels are defined that may contain multiple lighting channels.
For convenience, these pixels are also mapped to Scan Codes at the same time (if relevant).

The initial channel number is hardware specific and doesn't have to be in order.
While the pixel number is arbitrary and may be any number you like (stick with lower numbers for ideal packing).

\begin{lstlisting}
# Format
P[<pixel>](<ch 1>:<width>, ...) : <Scan Code>;

# Pixel 5, Channel 30, Single channel
# 8 bit width, Scan Code 0x13
P[5](30:8) : S0x13;

# Pixel 6, Channels 35-37, Triple channel
# 2 bit width, Misc Pixel
P[6](35:2, 36:2, 37:2) : None;
\end{lstlisting}


%% Animations
\section{Animations}
\label{sec:animations}

Animations are sets of changes applied to one of more mapped pixels over a period of time.
Values may be applied to single or sets of pixels by various means including setting, adding, subtracting, etc.
For convenience pixels are automatically grouped into layers if Scan Codes were used during the pixel mapping.

Each animation is given a label.
A label is an arbitrary text label (KLL string rules apply) used by KLL to identify the animation set.

\begin{lstlisting}
A[<label>] <= <modifiers>;
A[MyEyesAreBleeding] <= <modifiers>;
\end{lstlisting}

While defining the animation label, default animation modifiers can be applied.
Modifiers can include options such as the number of loops to run the animation, clock divider for animation speed or special hardware features the pixels may have (e.g.\ hardware fade).

The following modifiers are supported.

\begin{lstlisting}
loops:<num> # Specific number of loops
loop        # Loop infinitely
divshift:<num>
            # Animation frame divider shift
	    # Needs to match divmask
	    # e.g. 0x03 => 2
	    #      0x0F => 4
divmask:<num>
            # Animation frame divider mask
	    # # of loops for frame transition
	    # Must be contiguous
	    # e.g.  0x00, 0x01, 0x03, 0x0F
	    # *NOT* 0x02, 0x08, 0x24
start       # Start animation
pause       # Pause animation
stop        # Stop animation
single      # Play single animation frame
            #  then pause
pos:<frame> # Set animation frame number
pfunc:<name>(<args>)
            # Frame pixel fill function
	    # off
	    # interp
	    #  Interpolate between pixels
	    #  Slow, uses more CPU
	    # kllinterp
	    #  Pre-computed interpolation
	    #  Fast, uses more memory
ffunc:<name>(<args>)
            # Frame extension fill function
	    # off
	    # interp
	    #  Add frames dynamically
	    #  using interpolation
	    #  i.e. use defined as key-frames
	    #  Slow, uses more CPU
	    # kllinterp
	    #  Pre-computed interpolation
	    #  Fast, uses more CPU
replace:<type>
            # By default do not replace, just stack
	    # stack
	    #  Stack animation, do not replace
	    # basic
	    #  Replace if same index and trigger
	    # all
	    #  Replace if same index
	    # state
	    #  Replace if same index, trigger and
	    #  incoming state is press.
	    #  If state is release, stop.
	    # clear
	    #  Remove all animations in stack before
	    #  adding, regardless of index.
	    # clearactive
	    #  Remove all animations in stack before
	    #  adding, except for paused/soon paused
	    #  and do a replace state for the new
	    #  animation
\end{lstlisting}

An example modifier setup for an animation.

\begin{lstlisting}
# At keyboard start, loop animation 3 times
A[MyEyesAreBleeding] <= start, loop:3;
\end{lstlisting}

Using the defined logical pixels, each frame of the animation can be defined.
It is also possible to use ranges and pre-defined layers as groups of pixels within an animation set as a single colour.
When sets contain heterogenious sets of number of channels (e.g.\ RGB vs.\ monochrome) the first channel operations are used first and the extra are ignored for affected pixels.

Pixel values are changed using operator:value pairs.
The operator, if defined, indicates that the defined value modifies the current value rather than just overriding it.
When evaluating the animation stack, the stack is parsed top to bottom until a set (no operator) value is found.
This value is applied, then the above animations for that pixel are applied bottom to top.

The following operators are supported.

\begin{lstlisting}
<no op> # Set
+       # Add
-       # Subtract
+:      # Add no roll-over
-:      # Add no roll-over
<<      # Left shift
>>      # Right shift
\end{lstlisting}

Each frame of the animation must be defined individually describing what changes each frame makes to which pixels.
One or more pixels may be assigned to that animation frame.
It's also possible to use pixel groupings and ranges of pixels.

\begin{lstlisting}
# Basic animation frame syntax
A[<label>, <frame>] <= <pixels>;
# Animation "Bleed", frame 2
A[Bleeed, 2] <= <pixels>;

# Single pixel syntax
A[Bleeed, 2] <= P[<pixel>](<op><ch value>);
# Add 52 to first two channels of pixel 12
A[Bleeed, 3] <= P[12](+52, +52);

# Set two pixels to 20
A[Bleeed, 4] <= P[3](20), P[4](20);

# Set function layer 2 pixels to white in 24-bit RGB
A[Bleeed, 5] <= PL[2](255,255,255);

# Function layer and pixel
A[Bleeed, 6] <= PL[2](300,200,100), P[30](30);

# Ranges and lists of pixels
A[Bleeed, 7] <= P[1-20](40);
A[Bleeed, 8] <= P[1,4,6,8,9](0x25);
\end{lstlisting}


% Pixel Addressing
\subsection{Pixel Addressing}

Pixels, as part of an animation, have a few different types of addressing.
These make it easier to build animations more concisely and support a wide variety of keyboards with minimal/no modifications to the animation.

\begin{lstlisting}
P[2]          # Pixel index
P[r:3]        # Fill row
P[c:10]       # Fill column
P[r:3,c:10]   # Rectangle index
P[r:5%,c:10%]
              # Rectangle % index
P[i+2]        # Relative pixel index
P[r:i+3]      # Relative row fill
P[c:i-10]     # Relative column row fill
P[r:i-3,c:10] # Relative rectangle index
P[r:i+5%,c:i-3%]
              # Relative rectangle % index
U"A"          # USB Code index  (current layer)
S0x10         # Scan Code index (current layer)
\end{lstlisting}


%% Triggers
\section{Triggers}
\label{sec:Animation_Triggers}

Animations also generate a few different kinds of triggers.
These triggers can be used for anything from starting another animation (i.e.\ linked animations), custom capabilties, or even USB keyboard output.
The result syntax is identical to all other KLL triggers.

There are two types of triggers available: done \textbf{D} and repeat \textbf{R} triggers.
The end trigger, triggers when the animation finishes all scheduled loops.
This means that an infinitely running animation will never have a done trigger.
The repeat trigger is activated after each loop of an animation is finished.
Triggers do not go off if the animation is forceably ended by means of an alternate trigger, but if another trigger sets an infinitely running animation to a finite number of loops, then it may trigger an end trigger.

\begin{lstlisting}
# Done/Repeat trigger
A[LongAnim]    : <result>;

# Done trigger
A[LongAnim](D) : <result>;

# Repeat trigger
A[YourAnim](R) : <result>;
\end{lstlisting}

Per pixel triggers and animation frame triggers are not supported.
However it is possible to extend the spec to support this if there are enough requests.


%% Results
\section{Results}

Animations can also be dynamically controlled using a triggers.
The standard results mechanism is used to interact with the animations.

All animation modifiers may be used to control the animation.
See Section~\ref{sec:animations} for more details on each of the modifiers.

To be clear, it is completely valid to use an animation trigger to modify itself.

\begin{lstlisting}
<trigger> : A[Dynamo];
<trigger> : A[Dynamo](frame:3, interp:on);
<trigger> : A[Dynamo](stop);
\end{lstlisting}

Pixels and pixel groups are limited to channel setting modifiers and perhaps any additional hardware modifiers that may be available.

\begin{lstlisting}
# Operator on values to an 8-bit RGB pixel
<trigger> : P[23](+43,+21,-40);

# Set a pixel grouping layer 3 to 54
<trigger> : PL[3](54);
\end{lstlisting}


%% Pixel Positioning
\section{Pixel Positioning}

In order to support complex animation support, each pixel can be assigned a physical position.
This allows for turning sets of pixels into a display which can render basic fonts and images.

If the pixel was assigned to a Scan Code, then the pixel will inherit the positioning of the Scan Code key.
However, you can still override the pixel position (will not affect the Scan Code position).

Each key may be assigned the following options, by default they are set to 0.

\begin{itemize}
\item x, y, z positions (mm)
\item rx, ry, rz rotations (deg)
\end{itemize}

The units are in mm and degrees respectively.

\begin{lstlisting}
# Set the x position 20 mm and x rotation 15 deg
P[30] <= x:20,rx:15;
\end{lstlisting}

Currently the size of the pixel is not taken into account.
In general this should not matter for most purposes but it is possible to extend the spec to help deal with pixel shapes.


%%% Compilation %%%
\chapter{Compilation}

Compilation is relatively simple.
Reduce all triggers down to Scan Codes and all results into capabilities and USB Codes.
To combine keymaps, they must be stacked during compilation.
The first layout specified is the default.
Each layout layered on top takes precedence over the layout underneath of it.

There is no requirement to support online key remapping, EEPROM storage or post compilation key remapping (pre microcontroller flashing).

Keys that are not mapped should be ignored and not present.
Capabilities specified, but do not exist for a given keyboard are also ignored.

Warnings should be given about ambiguous ranges or no keymappings/trigger:result pairs are mapped.

Compilation should not succeed if there are missing mandatory variables or syntax problems.
Additionally, there cannot be any duplicate capabilities defined, whether they are ignored or not.


% USB Code Table
\newpage
\chapter{USB Code Table}
\label{chpt:USBCodeTable}

Following is a table of USB Codes.
For each USB Code there is at least a long and short name.
The names are case-insensitive.
Some USB Codes have multiple names for convenience.
Spaces and underscores both translate to underscore internally.

Please refer to the USB Keyboard HID spec for more details.
All event based key codes that do not correspond to a physical key are not included in the table.
For example, USB Code 0x01, which is a rollover error indicator.

Many USB Codes are recognized/implemented on the OS side.
Do not expect lesser known USB Codes to "just work".

\begin{ltable}{usbcodes}{ l | r | l }{Table of USB Codes}

\textbf{USB Code} & \textbf{Name} & \textbf{Alternate Name(s)} \\
\hline
\hline

\input{USBCodeTable} % See .tex file for data

\end{ltable}

As function keys are not part of the USB HID spec, they are difficult to reference using USB Codes.
To get around this, KLL reserves USB Codes 0xF0 to 0xFF for function keys.
It is possible (though unlikely) that a new USB HID spec may use this reserved space.
In that case, a new revision of KLL will be made to address this issue.
USB Codes 0xF0 to 0xFF are never sent via USB and are for internal processing only.

It is possible to use any arbitrary USB Code as a function key.
However, using something like a SysReq as your function key (in the default map) makes it unclear to other readers unless you explicitly comment this case in the .kll file.

\href{http://www.usb.org/developers/hidpage/Hut1_12v2.pdf}{USB Keyboard/Keypad HID Table} (Section 10, pg. 53).

\href{http://www.usb.org/developers/hidpage/Hut1_12v2.pdf}{http://www.usb.org/developers/hidpage/Hut1\_12v2.pdf}


% System Control Code Table
\newpage
\chapter{System Control Code Table}
\label{chpt:SysCodeTable}

Following is a table of USB System Control Codes.
These are useful for defining keys to signal the OS to power off, sleep or eject something.

Please refer to the USB HID spec for more details.
Do not expect your OS to recognize these codes.
Many of them are system specific.

\begin{ltable}{syscodes}{ l | r | l }{Table of System Control Codes}

\textbf{Sys Code} & \textbf{Name} & \textbf{Alternate Name(s)} \\
\hline
\hline

\input{SystemCodeTable} % See .tex file for data

\end{ltable}

\href{http://www.usb.org/developers/hidpage/Hut1_12v2.pdf}{System Control Table - Generic} (Section 4, pg. 26).

\href{http://www.usb.org/developers/hidpage/Hut1_12v2.pdf}{http://www.usb.org/developers/hidpage/Hut1\_12v2.pdf}


% Consumer Control Code Table
\newpage
\chapter{Consumer Control Code Table}
\label{chpt:ConsCodeTable}

Following is a table of USB Consumer Control Codes.
These are useful for defining media keys such as Next, Previous, Pause, etc.

Please refer to the USB HID spec for more details.
Do not expect your OS to recognize these codes.
Many of them are system specific.

\begin{ltable}{conscodes}{ l | r | l }{Table of Consumer Control Codes}

\textbf{Cons Code} & \textbf{Name} & \textbf{Alternate Name(s)} \\
\hline
\hline

\input{ConsumerCodeTable} % See .tex file for data

\end{ltable}

\href{http://www.usb.org/developers/hidpage/Hut1_12v2.pdf}{Consumer Control Table} (Section 15, pg. 75).

\href{http://www.usb.org/developers/hidpage/Hut1_12v2.pdf}{http://www.usb.org/developers/hidpage/Hut1\_12v2.pdf}


% LED Indicator Code Table
\newpage
\chapter{LED Indicator Code Table}
\label{chpt:LEDIndicatorCodeTable}
Following is a table LED Indicator Codes defined by the USB HID Spec.
On most OSs only NumLock, ScrollLock and CapsLock are used.
Even if the keyboard supports all of the LED Indicator Codes, unless the OS supports it, you will not be able to use it.

For completeness, the entire table is listed here.

\begin{ltable}{indicatorcodes}{ l | r | l }{Table of USB LED Indicator Codes}

\textbf{Indicator Code} & \textbf{Name} & \textbf{Alternate Name(s)} \\
\hline
\hline

\input{LEDIndicatorTable} % See .tex file for data

\end{ltable}

\href{http://www.usb.org/developers/hidpage/Hut1_12v2.pdf}{LED Table} (Section 11, pg. 61).

\href{http://www.usb.org/developers/hidpage/Hut1_12v2.pdf}{http://www.usb.org/developers/hidpage/Hut1\_12v2.pdf}


% Capabilities Table
\newpage
\chapter{Capabilities Table}
\label{chpt:CapabilitiesTable}

Following is a table of common capabilities.
Whenever defining new capabilities, make sure to check this table to make sure there are no namespace clashes.
The number beside each of the arguments specifies the size of the argument in bytes (e.g. 2 $\Rightarrow$ 16 bits).

\begin{ltable}{capabilityTable}{ l | l | l }{Table of Common Capabilities}

\textbf{Capability} & \textbf{Function} & \textbf{Argument(s)} \\
\hline
\hline

% Capability Table - Jacob Alexander 2014-2015

consCtrlOut & Output\textunderscore consCtrlSend\textunderscore capability & consCode : 2 \\
kbdProtocolBoot & Output\textunderscore kbdProtocolBoot\textunderscore capability & \\
kbdProtocolNKRO & Output\textunderscore kbdProtocolNKRO\textunderscore capability & \\
layerLatch & Macro\textunderscore layerLatch\textunderscore capability & layer : 2 \\
layerLock & Macro\textunderscore layerLock\textunderscore capability & layer : 2 \\
layerRotate & Macro\textunderscore layerRotate\textunderscore capability & previous : 1 \\
layerShift & Macro\textunderscore layerShift\textunderscore capability & layer : 2 \\
layerState & Macro\textunderscore layerState\textunderscore capability & layer : 2, state : 1 \\
noneOut & Output\textunderscore noneSend\textunderscore capability &  \\
rotate & Macro\textunderscore rotate\textunderscore capability & index : 1, rotation : 1 \\
sysCtrlOut & Output\textunderscore sysCtrlSend\textunderscore capability & sysCode : 1 \\
usbKeyOut & Output\textunderscore usbCodeSend\textunderscore capability & usbCode : 1 \\

 % See .tex file for data

\end{ltable}


% Locale Table
\newpage
\chapter{Locale Table}
\label{chpt:LocaleTable}

Following is a table of USB HID locale codes (i.e. \textbf{bCountryCode}'s).
Codes go from 0 to 255, most codes are reserved and unused.

\begin{ltable}{localeTable}{ l | l | l }{Table of USB HID Locales}

\textbf{Code} & \textbf{Name} & \textbf{Alternate Name(s)} \\
\hline
\hline

% Locale Table - Jacob Alexander 2018

0 & Undefined & Not Supported \\
1 & Arabic & \\
2 & Belgian & \\
3 & Canadian-Bilingual & \\
4 & Canadian-French & \\
5 & Czech Republic & \\
6 & Danish & \\
7 & Finnish & \\
8 & French & \\
9 & German & \\
10 & Greek & \\
11 & Hebrew & \\
12 & Hungary & \\
13 & International (ISO) & \\
14 & Italian & \\
15 & Japan (Katakana) & \\
16 & Korean & \\
17 & Latin America & \\
18 & Netherlands/Dutch & \\
19 & Norwegian & \\
20 & Persian (Farsi) & \\
21 & Poland & \\
22 & Portuguese & \\
23 & Russia & \\
24 & Slovakia & \\
25 & Spanish & \\
26 & Swedish & \\
27 & Swiss/French & \\
28 & Swiss/German & \\
29 & Switzerland & \\
30 & Taiwan & \\
31 & Turkish-Q & \\
32 & UK & \\
33 & US & \\
34 & Yugoslavia & \\
35 & Turkish-F & \\
36-255 & Reserved & \\

 % See .tex file for data

\end{ltable}
\href{http://www.usb.org/developers/hidpage/HID1_11.pdf}{HID Descriptor} (Section 6.2.1, pg. 23).

\href{http://www.usb.org/developers/hidpage/HID1_11.pdf}{http://www.usb.org/developers/hidpage/HID1\_11.pdf}


%%% Glossary %%%
\newpage
\begin{fglossary}
\begin{description}[itemsep=0em]

\glossent{Fall-through}{When a key is undefined on a particular layer, the key definition on the previously stacked layer will be used. Eventually the key definition will be set to using the default layer. If the None keyword is used, then the fall-through will stop and no action will take place.}
\glossent{Latch}{When referring to keyboards, a key function that is only enabled until the release of the next keypress.}
\glossent{Lock}{When referring to keyboards, a key function that is enabled until that key is pressed again (e.g. Caps Lock).}
\glossent{NKRO}{N-Key Rollover is the capability to press N number of keys at the same time on a keyboard and have them all register on the OS simultaneously.}
\glossent{Scan Code}{Row x Column code or native protocol code used by the keyboard.}
\glossent{Shift}{When referring to keyboards, a key function that is enabled while that key is held.}
\glossent{USB Code}{Keyboard Press/Release codes as defined by the USB HID Spec.}

\end{description}
\end{fglossary}


%%% TODO %%%
\newpage
\chapter{Future Considerations}

Possible features for future revisions of the KLL spec.
If the feature is in this list, there is no guarantee that it will be in the specified version series.

\section{Possible 0.6+}

\begin{itemize}
\item Add capabilities descriptions
\item Mouse control
\item Joystick control
\item Unicode output
\item More concise way to indicate capability defaults
\item Analog velocity and acceleration detection (optional, depends on hardware)
\end{itemize}


\end{document}

